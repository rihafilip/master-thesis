%---------------------------------------------------------------
\chapter*{Introduction}\addcontentsline{toc}{chapter}{Introduction}\markboth{Introduction}{Introduction}
%---------------------------------------------------------------
\setcounter{page}{1}

\todo{explain more the feedback and its role for compilaiton}

Many modern dynamic languages are executed on a virtual machine (VM). In order to speed up the execution of programs, VMs traditionally include a Just-in-Time (JIT) compilers, allowing them to improve the performance of frequently executed pieces of programs by compiling them to native code. To further enhance the performance, most modern VMs record information about the runtime (called feedback), allowing them to predict future behaviour and optimize the compilation even more, resulting in faster execution. This is based on the thesis that what is past is prologue. However, over time, the recorded information tends to become less precise, resulting in more general assumptions and slower code down the line. This trend is known as feedback pollution.

R is a high-level programming language specialized for statistical computing and data visualization. It is dynamically typed with function and object-oriented patterns, and features reflection and lazy evaluation. GNU-R, the reference implementation of R, contains both an abstract syntax tree (AST) interpreter and a bytecode JIT compiler and interpreter. Ř is an alternative JIT compiler for GNU-R. It uses feedback information from an interpreter to speculatively optimize the native compilation. However, the inner workings of Ř are sort of a black box, as there is no simple way to trace its functionality. \todo{expand}

The main goal of this thesis is to implement a tool that would allow us to observe and analyze the behavior of the Ř compiler, allowing us to better understand the Ř internals and opening up possibilities for further research and analysis. Concretely, in this thesis, we are interested in understanding the feedback pollution as it happens in R.

In chapter 1, we introduce the necessary background information, including the R language, GNU-R virtual machine implementation, and the Ř compiler structure. Chapter 2 delves into the design and implementation of the recording tool, as well as its impact on the Ř codebase. Chapter 3 outlines the research done on the feedback pollution, which was possible thanks to the recording tool. Chapter 4 then further analyzes how the feedback information is used (or not used) during compilation, observing feedback patterns in the compiler and connecting them to the feedback pollution.

