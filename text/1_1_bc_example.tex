\begin{listing}[p]
	\centering
	\begin{adjustwidth}{-1cm}{-1cm}
		\begin{minipage}{0.18\textwidth}
			\begin{minted}[fontsize=\small,linenos=false]{R}
f <- function(x) {
  for (i in 1:10) {
    if (i + 2 > 1) {
      g(x)
    }
  }
  g(x)
}
    \end{minted}
			\subcaption{\centering R code}\label{lst:bc-example-r}
		\end{minipage}
		\hfill
		\begin{minipage}{0.80\textwidth}
			\begin{minipage}{0.18\textwidth}
				\begin{minted}[fontsize=\small,linenos=false]{text}
Code:
  1 LDCONST 1
  3 STARTFOR 4 3 30
  7 GETVAR 3
  9 LDCONST 5
 11 ADD 6
 13 LDCONST 7
 15 GT 8
 17 BRIFNOT 9 28
 20 GETFUN 10
 22 MAKEPROM 12
 24 CALL 11
 26 GOTO 29
 28 LDNULL
 29 POP
 30 STEPFOR 7
 32 ENDFOR
 33 POP
 34 GETFUN 10
 36 MAKEPROM 12
 38 CALL 11
 40 RETURN
        \end{minted}
			\end{minipage}
			\hfill
			\begin{minipage}{0.60\textwidth}
				\begin{minted}[fontsize=\small,linenos=false]{text}
Constant pool:
0:
 language {  for (i in 1:10) {; i..
1:
 int [1:10] 1 2 3 4 5 6 7 8 9 10
2:
 language 1:10
...
12:
 Promise 0:
  Code:
    1 GETVAR 0
    3 RETURN
  Constant pool:
  0:
   symbol x
  1:
   language g(x)
  2:
   'expressionsIndex' int [1:4] N..
13:
 'expressionsIndex' int [1:41] NA..
        \end{minted}
			\end{minipage}
			\subcaption{\centering GNU-R code}\label{lst:bc-example-gnur}
		\end{minipage}
	\end{adjustwidth}
	\par\vspace{2mm}\par
	\begin{minipage}{\textwidth}
		\centering
		\begin{minipage}{0.47\textwidth}
			\begin{minted}[fontsize=\small,linenos=false]{\rirlexer}
0:
      0   push_  1
      5   visible_
      6   force_
      7   push_  10
     12   visible_
     13   force_
     14   ; :(1, 10)
          colon_input_effects_
     15   pop_
     16   swap_
     17   colon_cast_lhs_
     18   [ <?> ] Type#0
     23   ensure_named_
     24   swap_
     25   colon_cast_rhs_
     26   ensure_named_
     27   [ <?> ] Type#1
     32   dup2_
     33   ; NULL
          le_
     34   [ _ ] Test#0
     39   brfalse_  1
     44   push_  1L
     49   br_  2
      \end{minted}
		\end{minipage}
		\hfill
		\begin{minipage}{0.47\textwidth}
			\begin{minted}[fontsize=\small,linenos=false]{\rirlexer}
1:
     54   push_  -1L
     ...
7:
    287   popn_  3
    292   ldfun_  g
    297   [ 0, <0>, valid  ] Call#2
    302   mk_promise_  2
    307   ; g(x)
          call_  1
    324   [ <?> ] Type#11
    329   ret_

[Prom (index 0)]
0:
      0   ldvar_  x
      5   [ <?> ] Type#5
     10   ret_

[Prom (index 1)]
0:
      0   ldvar_  x
      5   [ <?> ] Type#9
     10   ret_
     ...
      \end{minted}
		\end{minipage}
		\subcaption{\centering RIR code}\label{lst:bc-example-rir}
	\end{minipage}
	\caption{A truncated example of generated GNU-R and RIR bytecodes, full code in appendix \ref{ch:appendix-bc}}\label{lst:bc-example}
	% \ref{ch:appendix-bc}
\end{listing}
